\documentclass[11pt]{article}
\usepackage[a4paper, margin=2.54cm]{geometry}
\usepackage[utf8]{inputenc}
% \usepackage[spanish, mexico]{babel}
\usepackage[spanish]{layout}
\usepackage[article]{ragged2e}
\usepackage{textcomp}
\usepackage{amsmath}
\usepackage{amssymb}
\usepackage{amsfonts}
\usepackage{fancyvrb}
\usepackage{xcolor}
% \usepackage{proof}

% \setlength{\parindent}{5pt}

\title{
    Trabajo Práctico N° 4 \\
    \large Análisis de Lenguajes de Programación}
\author{Mellino, Natalia \and Farizano, Juan Ignacio}
\date{}

\begin{document}
\maketitle
\noindent\rule{\textwidth}{1pt}

\section*{Ejercicio 1) a.}

\subsection*{Monad.1}

Queremos probar \texttt{return x >>= f = f x}

\begin{Verbatim}[commandchars=\\\{\}]
    \textcolor{black}{return x >>= f}
    \textcolor{red}{= <state.1>}
    \textcolor{black}{State (\textbackslash s -> (x :!: s)) >>= f}
    \textcolor{red}{= <state.2>}
    \textcolor{black}{State (\textbackslash s -> let (v :!: s') = runState (State (\textbackslash s -> (x :!: s))) s
                                   in runState (f v) s')}
    \textcolor{red}{= <def runState>}
    \textcolor{black}{State (\textbackslash s -> let (v :!: s') = (\\s -> (x :!: s)) s
                                                 in runState (f v) s')}
    \textcolor{red}{= <beta-redex>}
    \textcolor{black}{State (\textbackslash s -> let (v :!: s') = (x :!: s) in runState (f v) s')}
    \textcolor{red}{= <def Let>}
    \textcolor{black}{= State (\textbackslash s -> runState (f x) s)}
    \textcolor{red}{= < eta-redex >}
    \textcolor{black}{= State (runState (f x))}
    \textcolor{red}{= < State . runState = Id >}
    \textcolor{black}{f x}
\end{Verbatim}

\subsection*{Monad.2}

Queremos probar: \texttt{t >>= return     = t}

\begin{Verbatim}[commandchars=\\\{\}]
    \textcolor{black}{t >>= return}
    \textcolor{red}{= < t :: State a >}
    \textcolor{black}{State g >>= return}
    \textcolor{red}{= <state.2>}
    \textcolor{black}{State (\textbackslash s -> let (v :!: s') = runState (State g) s in runState (return v) s')}
    \textcolor{red}{= <def runState>}
    \textcolor{black}{State (\textbackslash s -> let (v :!: s') = g s in runState (return v) s')}
    \textcolor{red}{= <state.1>}
    \textcolor{black}{= State (\textbackslash s -> let (v :!: s') = g s in runState (State (\textbackslash s -> (v :!: s))) s')}
    \textcolor{red}{= <def runState>}
    \textcolor{black}{= State (\textbackslash s -> let (v :!: s') = g s in (\textbackslash s -> (v :!: s)) s')}
    \textcolor{red}{=<beta-redex>}
    \textcolor{black}{= State (\textbackslash s -> let (v :!: s') = g s in (v :!: s'))}
    \textcolor{red}{= <def let>}
    \textcolor{black}{= State (\textbackslash s -> g s)}
    \textcolor{red}{= < eta-redex >}
    \textcolor{black}{= State g}
\end{Verbatim}

\subsection*{Monad.3}

Queremos probar: \texttt{(t >>= f) >= g   = t >>= (\textbackslash x -> f x >>= g)}

\begin{Verbatim}[commandchars=\\\{\}] 
    (t >>= f) >>= g
    \textcolor{red}{= < t :: State a -> t = State h >}
    \textcolor{black}{(State h >>= f) >== g}
    \textcolor{red}{= <state.2>}
    State (\textbackslash s -> let (v :!: s') = runState (State h) s
                  in runState (f v) s') >>= g
    \textcolor{red}{= <def runState>}
    State (\textbackslash s -> let (v :!: s') = h s 
                  in runState (f v) s') >>= g
    \textcolor{red}{= <state.2>}
    State (\textbackslash z -> let (b :!: z') = runState (State (\textbackslash s -> let (v :!: s') = h s
                                                           in (runState (f v)) s')) z
                  in runState (g b) z')
    \textcolor{red}{= <def runState>}
    State (\textbackslash z -> let (b :!: z') = (\textbackslash s -> let (v :!: s') = h s 
                                           in (runState (f v)) s') z
                  in runState (g b) z')
    \textcolor{red}{= <beta-redex>}
    State (\textbackslash z -> let (b :!: z') = (let (v :!: s') = h z 
                                    in (runState (f v)) s')
                  in runState (g b) z')
    \textcolor{red}{= < prop let | x = (b :!: z'), y = (v :!: s'), f = h z, h y = (runState (f v)) s', g x = runState (g b) z'>}
    State (\textbackslash z -> let (v :!: s') = h z in (let (b :!: z') = (runState (f v)) s' 
                                           in runState (g b) z'))
    \textcolor{red}{= < beta-expand >}
    = State (\textbackslash z -> let (v :!: s') = h z in (\textbackslash s -> let (b :!: z') = (runState (f v)) s
                                                    in runState (g b) z') s')
    \textcolor{red}{= < def runState >}
    State (\textbackslash z -> let (v :!: s') = h z 
                  in runState (State (\textbackslash s -> let (b :!: z') = (runState (f v)) s
                                             in runState (g b) z')) s')
    \textcolor{red}{= < beta-expand >}
    State (\textbackslash z -> let (v :!: s') = h z 
                  in runState ((\textbackslash x -> (State (\textbackslash s -> let (b :!: z') = (runState (f x)) s
                                                      in runState (g b) z'))) v) s')
    \textcolor{red}{= < state.2 >}
    State (\textbackslash z -> let (v :!: s') = h z 
                  in runState ((\textbackslash x -> f x >>= g) v) s')
    \textcolor{red}{= < def runState >}
    State (\textbackslash z -> let (v :!: s') = runState (State h) z 
                  in runState ((\textbackslash x -> f x >>= g) v) s')
    \textcolor{red}{= < state.2 >}
    (State h) >>= (\textbackslash x -> f x >>= g)
    \textcolor{red}{= < t :: State a -> t = State h >}
    t >>= (\textbackslash x -> f x >>= g)
\end{Verbatim}

\end{document}